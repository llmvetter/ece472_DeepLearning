\documentclass[conference]{IEEEtran}
\IEEEoverridecommandlockouts
% ! LTeX: language=en-US
\usepackage{cite}
\usepackage{url}
\usepackage{amsmath,amssymb,amsfonts}
\usepackage{algorithmic}
\usepackage{graphicx}
\usepackage{textcomp}
\usepackage{xcolor}
\def\BibTeX{{\rm B\kern-.05em{\sc i\kern-.025em b}\kern-.08em
    T\kern-.1667em\lower.7ex\hbox{E}\kern-.125emX}}
\begin{document}

\title{Ethical Arguments for the use of AI-driven Monitoring Systems for Remote Employees}

\author{\IEEEauthorblockN{Lennart Vetter}
\IEEEauthorblockA{\textit{Albert Nerken School of Engineering (The Cooper Union)} \\
New York City, USA \\
lennart.vetter@cooper.edu}}
\maketitle

\begin{abstract}
As the adoption of remote work environment increases, so does the need of 
adequate tools to monitor these spaces. This paper presents ethical justifications for 
the use of AI-driven Monitoring Systems that incorporate employee biometric data to 
monitor their productivity, arguing that these systems are guided by the moral 
imperative of improving employee welfare, increasing workplace fairness and safety
and having a positive business impact.
\\
\end{abstract}

\begin{IEEEkeywords}
Algorithmic Control, Surveillance, Machiavellianism, Authoritarianism, Orwellianism.
\end{IEEEkeywords}

\section{The Need for Employee Monitoring} % INTRO

The recent shift to remote work has provided employees with greater workplace flexibility and access to 
more resources and opportunities. Remote offers many benefits for employees such as saving valuable
time commuting, enhancing the work-life balance and increasing work flexibility.
However, these advantages come with several unique challenges: One of the primary difficulties 
remote workers face in these online environments is the lack of interaction with their supervisor 
or colleagues. The absence of a personal connection in virtual work environments and the 
blurring of work and non-work boundaries can negatively impact the motivation of employees, 
decrease social interactions \cite{Yang2022} which contributes to a sense of isolation, 
a lack of motivation and increased procrastination \cite{Bloom2014}.
AI-driven Monitoring Systems describe a set of tools which autonomously track biometric data
of employees such as facial expressions, eye movements, head position and body posture as 
well as behavioral data like keystrokes and mouse movements in an effort to grant real time and 
fine-grained insights into workers' productivity and health metrics. 

This systematic review lays out the benefits of implementing AI driven Employee Monitoring Systems.
These systems provide data driven support for employee well-being, increase and enforce employee 
workplace safety, optimize organizational resources and enforce equal accountability, benefitting 
both the employer and the employee which serves as a moral justification for the use 
of these systems.

\section{New Environments, New Challenges} % DISTRACTION / ASSISTANCE

Remote environments differ significantly from traditional office environments. Many factors that 
emanate from the physical work environment such as the presence of coworkers,
adequate work equipment and a dedicated work environment effect workers' productivity and 
are out of reach for the employer's influence. Furthermore, remote environments give rise 
to distractions from both external factors like environmental disturbances and internal 
factors like low levels of engagement.

\subsection{Staying Focused}

These distractions include constant private phone notifications, emails, phone calls or other external interruptions. 
In remote work settings, these distractions are of particular concern as the boundary between work related and 
non-work related tasks blurs. Multitasking, such as reacting to and taking care of private errands or interacting 
with social media, significantly reduces cognitive efficiency \cite{Blasiman2018}.
Remote workers are particularly vulnerable to distractions that significantly impact their focus and 
overall performance outcomes. As \cite{Altmann2014} demonstrated, momentary interruptions, even as 
brief as a few seconds, can significantly increase error rates by disrupting the cognitive
processes required to maintain task and sequence continuity. As a result, maintaining concentration 
and staying focused on a task becomes a major challenge in remote work settings. Understanding, 
detecting and predicting when remote workers become distracted is therefore of high importance.

Physiological and biometric signals such as cognitive load, blinking of the eyes, facial expression and 
head and body pose have proven highly valuable in predicting attention levels, distractions or 
measuring the current cognitive load. When integrating this data into analytics tools, more 
comprehensive and detailed insights can be obtained.
Simply by observing the head position, \cite{Becerra2025} was able to predict phone distractions 
with 87\% accuracy. Similarly, \cite{Daza2024} showed that webcam data can be successfully leveraged 
to accurately predict high and low attention levels by tracking eye blinking and facial expressions. 
A different study showed that by simply measuring body posture and facial expression using a
webcam in combination with machine learning, it was possible to predict whether participants 
are concentrated with more than 90\% recall \cite{Betto2022}.
Another study by \cite{Kaczorowska2021} showed, that using features extracted from eye tracking
like fixation rates and pupil diameter yields promising results in predicting cognitive 
workload levels with models achieving up to 92\% accuracy. 
Understanding these different performance metrics such as attention levels and cognitive workload 
levels is of great importance in analyzing human mental fatigue and in assessing the performance 
on intellectual tasks. Biometric data can further be leveraged to "identify sequential interaction 
patterns associated with moments of distraction" \cite{Rodriguez2023}, offering the unique 
opportunity to optimize the order of sequential tasks to be performed by employees. 

These results show that tracking biometric data is an effective way to measure employee task 
engagement. This which opens the door to implement intelligent systems that assist employees 
in their work instead of surveilling them. Such Assistant Systems work together the employees, 
assisting them by providing real time objective data and guidance to counteract 
cognitive challenges like distractions and mental fatigue as well as allowing for 
personalized feedback. These assistant systems would be able to recommend sequential 
tasks based on the employee's current cognitive load and fatigue level thus maximizing 
productivity while retaining attention as well as minimizing distractions and supporting the 
employee in maintaining focus. Hence, the moral justification arises from the supportive nature 
of these systems.


\subsection{Improving Workplace Safety} % SAFETY

Assisting the employee in terms of productivity is not the only capability of modern monitoring systems. 
When human supervision is absent, these systems watch out for the employees by utilizing the collected 
biometric and behavioral data to provide real time risk mitigation and personalized workload management. 
Physical safety of employees is the primary ethical and legal obligation of every employer. 
While this might be less of a concern in white collar remote work, in blue collar industries 
like the Construction, Remote Monitoring Systems proof to be incredibly helpful, increasing employee 
safety in various ways.

Real-time location tracking of employees, which is particularly useful in environments with reduced visibility,
can be used to prevent severe injury by avoiding worker collision with heavy machinery or vehicles in 
motion \cite{Arboleya2021}. Moreover, Computer Vision Systems are able to track if workers are utilizing
adequate safety equipment like wearing helmets or masks \cite{Das2022} for the work they are performing. 
Mobility related work heavily benefits from using behavioral and biometric data such as 
eye motion tracking, blink frequency, facial expressions and head movement in combination with Neural 
Networks to predict driver drowsiness with high degrees of accuracy, preventing sleepiness induced 
accidents \cite{Shaik2023}.

Only a healthy and uninjured employee can do his job in a productive manner. 
By using biometric data, these Remote Monitoring Systems can actively improve safety of employees in 
blue collar work environments and can also ensure that government imposed safety regulations are met.
As it is a company's highest ethical commitment to protect workers and ensure their physical safety, 
the use of these systems are not only morally justified but a moral obligation arises for the employer 
to use these tools for the benefit of his employees.

\subsection{Employee Health as a Top Priority} % HEALTH

While safety addresses the workplace as a whole, individual employee health
also benefits tremendously from Remote Monitoring Systems.
These systems promise to handle physical demands for each worker in a personalized manner by leveraging
individual biometric data: Wearable technologies can estimate individual activity levels through proxies 
like heart rate, which is a physiological variable that has been extensively used to quantify physical 
demands because of its association with cardiovascular loads \cite{Arias2023}. This essential information 
can then be used to manage physical demands within acceptable limits for each individual. 
Consequently, this data can be utilized to provide personalized alerts to the employee or 
manager, promoting mandatory pauses or task reallocation based on individual physiological limits.
Especially in construction work, work fatigue leads to poor judgement, increased risk of injuries, 
decreased productivity and lowers quality of work \cite{Abdelhamid2002}. 
Since physical fatigue is a predominant risk factor for injuries and illnesses in the 
construction industry, it is essential to monitor fatigue to reduce its adverse effects and 
prevent long-term health problems. This can be achieved by tracking biometric data like the aerobic 
fatigue threshold \cite{Bangaru2022}.

Monitoring systems are not only useful for real-time applications but also provide data for mitigating 
long term occupational injury risk. By accumulating biometric data from different employees over 
time, insights can be gained in which behavioral patterns predict injury risk and consequently 
conclusions can be drawn to improve long-term workspace safety procedures and 
requirements \cite{Khairuddin2022}.

Remote Monitoring Systems therefore go well beyond supervision and proactively promote employee health. 
Improving employee wellbeing and physical health lays the moral foundation for the use of this technology.

\subsection{Making Informed Decisions} % FAIRNESS AND BUSINESS ASPECTS

Beyond improving workplace safety, preventing employee injuries and preserving their health as well as 
assisting employees in staying focus and avoiding distractions, these systems also tremendously
benefit the employer. Fine-grained performance insights and data allows managers to quickly 
identify company inefficiencies. Managers are able to see which tasks employees are currently 
tackling and precisely how much time has been spent on every task. Knowing how current company 
resources are utilized opens the opportunity to channel them differently once inefficiencies are 
discovered.
Anecdotally, Ivan Petrovic, CEO and Founder of Insightful, told the story of how a client was able 
to save \$2 Million dollars by discovering that none of its employees used an expensive piece 
of software \cite{Piltch2025}. Similarly, tracking remote employee, the employer can quickly 
determine who is being underutilized and who does too much work. By providing objective metrics, 
Monitoring Systems promote fairness and equality among employees, allowing for an equal balance 
of tasks and workload across all team members which adheres to the moral principle of fairness of
contribution. Gained insights also allow managers to make informed decisions on who to promote and 
to whom assign specific, high stakes projects based on the performance measures that have 
been gathered per employee. Leveraging biometric data for promotions and assignments fosters a 
culture of objective and transparent decisions and fairness, moving away from subjective bias 
and management favoritism. Perceived fairness correlates with a range of positive employee outcomes, 
such as enhanced job satisfaction, increased trust, and greater engagement \cite{Jo2025}.

Remote Employee Productivity Monitoring can also serve a dual purpose of improving cyber-security 
by using the collected biometric and behavioral data such as keystrokes, eye motion, head motion 
or mouse tracking for continuous biometric authentication \cite{Boshoff2025}, flagging unusual 
or non-work related activities that could pose a risk to company data integrity. 

Consequently, the use of Remote Employee Productivity Monitoring can be very beneficial to 
the employer. If a company is unable to ensure company efficiency and security, it might be 
forced to end remote work policies which would be detrimental to employees who rely on that 
flexibility. It is therefore necessary to ensure the viability of the company as a prerequisite 
for employment.

\section{Debunking the Orwellian Concern}
Privacy oriented employees may raise valid ethical concerns regarding the idea of having AI driven
Monitoring Systems track their biometric data during work hours. The argument states that such 
systems manifest an unacceptable invasion into employee privacy and create a monitoring culture 
that goes well beyond work related data collection. Consequently, implementing the discussed Monitoring
Systems naively would go against ethical principle of respecting the individuals' autonomy and 
control over its personal information thus violating the moral obligation of data privacy.
The following paragraph shall address this common concern by providing three foundational
principles that shall guide the integration of AI-diven Monitoring Systems.

\subsection{Proportionality}
Monitoring should not extant beyond work-related activities. This ensures intrusion in minimized 
and the amount of data collection done is proportionate to its legitimate goal. Therefore, 
employees shall only be monitored during work hours and monitoring shall strictly be limited to 
company owned devices. By limiting the monitoring scope, it can be ensured that data tracked, 
collected and analyzed is work related only whereby privacy intrusion is minimized.

\subsection{Transparency}
System transparency is the most important principle towards ethical data processing. 
Transparency builds trust by giving both parties a reason to believe the other is looking 
out for its best interests \cite{Flynn2024}. Hence, openly communicating what data is collected, 
how it is processed and used and for which purpose the data is collected and analyzed should be of 
high importance.

\subsection{Mutual Benefit}
Monitoring Systems are not designed to unilaterally profit the employer and rather are 
fundamentally rooted in providing mutual benefit to both the employer and employee. The collected 
data serves to actively improve the work environment but enhancing safety and preventing injuries. 
These systems also actively contribute to boost employee productivity and minimizing distractions.
Furthermore, the ability to maintain remote work policies is based on the employer's assurance of
consistent performance and engagement. Monitoring provides the necessary security as well as
verifiable performance metrics in order to keep flexible work contracts in place. If no such 
security is given, then the employer might be forced to revoke remote work policy, negatively impacting
work flexibility for employees.
\\

By respecting the outlined morel principles of Proportionality, Transparency and Mutual Benefit 
the moral obligation to data privacy can be upheld.

\section{Conclusion}
This paper outlined the vast benefits and the guiding ethical principles for the 
integration off Remote Monitoring Systems, arguing that these systems promote better 
workplace safety and fairness, increase productivity and enhance employee wellbeing. 

It was demonstrated that these systems adequately address the unique challenges 
of remote work environments, serving as proactive assistant to the employee, 
minimizing distractions, boosting productivity and supporting overall health and safety. 
Furthermore, they allow employers to optimize resource allocation and strengthen cybersecurity 
which is crucial to maintain a remote work policy. This paper also addressed common employee 
concerns with respect to data privacy and data ownership and puts forward three fundamental 
principles to mitigate these concerns.

Ultimately, monitoring the digital workspace shall be analogous to a supervisor observing the 
physical workspace and the change in tooling represents a necessary evolution in workplace management
in accordance with modern flexibility demands of employees and to uphold ethical principles 
and moral obligations in a distributed work environment. Just like physical supervisor, 
these systems will check on presence, proper procedures and safety but also offer many more benefits 
due to their ability to leverage biometric data which grants insights of unprecedented depth.


\newpage

\begin{thebibliography}{00}
\bibitem{Piltch2025} A. Piltch, ``Bossware booms as bots determine wether you're doing a good job'' The Register, Nov. 23, 2025. [Online]. Available: \url{https://www.theregister.com/2025/11/23/bossware_monitor_remote_employees}
\bibitem{Becerra2025} Alvaro Becerra, Roberto Daza, Ruth Cobos P{\'e}rez, Aythami Morales, Mutlu Cukurova, and Julian Fi{\'e}rrez, ``AI-based Multimodal Biometrics for Detecting Smartphone Distractions: Application to Online Learning,'' {\it ArXiv}, 2025, vol. abs/2506.17364.
\bibitem{Altmann2014} Altmann, Erik M. et al. “Momentary interruptions can derail the train of thought.” Journal of experimental psychology. General 143 1 (2014): 215-26.
\bibitem{Blasiman2018} Blasiman, Rachael N. et al. “Distracted Students: A Comparison of Multiple Types of Distractions on Learning in Online Lectures.” Scholarship of Teaching and Learning in Psychology 4 (2018): 222–230.
\bibitem{Betto2022} Betto, Iku et al. “Distraction detection of lectures in e-learning using machine learning based on human facial features and postural information.” Artificial Life and Robotics 28 (2022): 166 - 174.
\bibitem{Daza2024} Daza, Roberto et al. “DeepFace-Attention: Multimodal Face Biometrics for Attention Estimation With Application to e-Learning.” IEEE Access 12 (2024): 111343-111359.
\bibitem{Rodriguez2023} Rodríguez, Andrés Ovidio Restrepo et al. “Application of learning analytics for sequential patterns detection associated with moments of distraction in students in e‐learning platforms.” Computer Applications in Engineering Education 32 (2023): n. pag.
\bibitem{Kaczorowska2021} Kaczorowska, Monika et al. “Interpretable Machine Learning Models for Three-Way Classification of Cognitive Workload Levels for Eye-Tracking Features.” Brain Sciences 11 (2021): n. pag.
\bibitem{Arboleya2021} A. Arboleya, J. Laviada, Y. Álvarez-López and F. Las-Heras, "Real-Time Tracking System Based on RFID to Prevent Worker–Vehicle Accidents," in IEEE Antennas and Wireless Propagation Letters, vol. 20, no. 9, pp. 1794-1798, Sept. 2021, doi: 10.1109/LAWP.2021.3097136. keywords: {Safety;Head;Linear antenna arrays;RFID tags;Antenna radiation patterns;Receiving antennas;Radar tracking;Industrial environments;radio frequency identification (RFID);receiver signal strength (RSS);real-time tracking},
\bibitem{Arias2023} Arias O, Groehler J, Wolff M, Choi SD. Assessment of Musculoskeletal Pain and Physical Demands Using a Wearable Smartwatch Heart Monitor among Precast Concrete Construction Workers: A Field Case Study. Applied Sciences. 2023; 13(4):2347. https://doi.org/10.3390/app13042347
\bibitem{Khairuddin2022} Khairuddin, Mohamed Zul Fadhli et al. “Occupational Injury Risk Mitigation: Machine Learning Approach and Feature Optimization for Smart Workplace Surveillance.” International journal of environmental research and public health vol. 19,21 13962. 27 Oct. 2022, doi:10.3390/ijerph192113962
\bibitem{Bangaru2022} Bangaru, Wang, Aghazadeh (2022). Automated and Continuous Fatigue Monitoring in Construction Workers Using Forearm EMG and IMU Wearable Sensors and Recurrent Neural Network. Sensors, 22(24), 9729. https://doi.org/10.3390/s22249729
\bibitem{Das2022} Das Prangon, Abhi Sarafat, Suhreed F., Uddin Jalal, Jahan Shahriar, Lara Roknuzzaman, Kashem Md. (2022). An Intelligent Industrial Safety and Health Monitoring System for Industry 4.0. 
\bibitem{Shaik2023} Md. Ebrahim Shaik, "A systematic review on detection and prediction of driver drowsiness", Transportation Research Interdisciplinary Perspectives 21 (2023), ISSN 2590-1982, https://doi.org/10.1016/j.trip.2023.100864.
\bibitem{Abdelhamid2002} Abdelhamid, T. S., and J. G. Everett. (2002). Physiological Demands during Construction Work. \textit{Journal of Construction Engineering and Management}, 128(5), 427--437.
\bibitem{Boshoff2025} Dutliff Boshoff, Gerhard P. Hancke, "A classifications framework for continuous biometric authentication (2018–2024)", Computers \& Security V. 150 (2025), https://doi.org/10.1016/j.cose.2024.104285.
\bibitem{Flynn2024} J. Flynn, S. Cantrell, D. Mallon, L. Kirby, ``The Transparency Paradox: Could less be more when it comes to trust'' Deloitte Insights, 05. Feb 2024. [Online]. Available: \url{https://www.deloitte.com/us/en/insights/topics/talent/human-capital-trends/2024/transparency-in-the-workplace.html}
\bibitem{Bloom2014} Bloom, Nicholas et al. “Does Working from Home Work? Evidence from a Chinese Experiment.” The Quarterly Journal of Economics 130 1 (2014): 165-218.
\bibitem{Yang2022} Yang, L., Holtz, D., Jaffe, S. et al. The effects of remote work on collaboration among information workers. Nat Hum Behav 6, 43–54 (2022). https://doi.org/10.1038/s41562-021-01196-4
\bibitem{Jo2025} Jo H, Shin D. The impact of recognition, fairness, and leadership on employee outcomes: A large-scale multi-group analysis. PLoS One. 2025 Jan 9;20(1):e0312951. doi: 10.1371/journal.pone.0312951. PMID: 39787185; PMCID: PMC11717283.
\end{thebibliography}

\vspace{12pt}

\end{document}
