\documentclass[conference]{IEEEtran}
\IEEEoverridecommandlockouts
% The preceding line is only needed to identify funding in the first footnote. If that is unneeded, please comment it out.
%Template version as of 6/27/2024

\usepackage{cite}
\usepackage{url}
\usepackage{amsmath,amssymb,amsfonts}
\usepackage{algorithmic}
\usepackage{graphicx}
\usepackage{textcomp}
\usepackage{xcolor}
\def\BibTeX{{\rm B\kern-.05em{\sc i\kern-.025em b}\kern-.08em
    T\kern-.1667em\lower.7ex\hbox{E}\kern-.125emX}}
\begin{document}

\title{Ethical Arguments for the use of AI-driven productivity monitoring of remote employees}

\author{\IEEEauthorblockN{Lennart Vetter}
\IEEEauthorblockA{\textit{Albert Nerken School of Engineering (The Cooper Union)} \\
New York City, USA \\
lennart.vetter@cooper.edu}}
\maketitle

\begin{abstract}
This paper highlights the importance of remote worker productivity monitoring using AI-driven biometric and behavioral data like key strokes, eye movement and mouse movements. Leveraging these data sources yield improved productivity, increases safety and grants more fine grained control over task distribution to the employer.
\end{abstract}

\begin{IEEEkeywords}
Algorithmic Control, Surveillance, Machiavellianism, Authoritarianism, Distopia.
\end{IEEEkeywords}

\section{Introduction}
Monitoring productivity is an essential and vital part of every company. With the increase of remote work ever since the pandemic, new challenges have occured to ensure worker productivity.
Remote productivity monitoring deals with tracking employee activity during working hours and checks if they are doing the tasks they are supposed to be doing. These tools include time-tracking software, computer and email monitoring as well as 
productivity monitoring software tracking biometric and behavioral data like eye-movements, key strokes and mouse movements to grant real time and fine grained insights into workers behavior. This systematic review lays out the benefits of implementing these
systems and how these systems can be leveraged to increase worker productivity.

\section{Body}

Productivity monitoring of remote employees offers a multitude of benefits.

\subsection{Improved Resource Utilisation}
Efficient and exhaustive resource utilisation is foundational for efficent companies. Knowing where to allocate the given resources builds upon knowledge of how teh company is currently using them. Productivity Monitoring offers finegrained insights into which tasks
employees are currently tackling, how much time has been spend on every taks. These fine grained insights provide an unprecidented opportunity to discover ressource inefficiencies. Anectodaly, Ivan Petrovic, CEO and Founder of insughtful, told the story of how a client 
was able to save \$2 Milion dollars by discovering that non of it's eployees used an expensive piece of software \cite{b1}.

Furthermore, Productivity Monitoring Software helps to allocate Human Ressources in a more meaningful way: By tracking each remote employee, the employer can quickly determine who is being underutilized and who does too much work. This helps to helps to balance tasks across
team members, ensuring every employee contributes equally to the success of the company.

\subsection{Deeper Understanding of Employee Performance}
TBD

\subsection{Making Informed Decisions}
TBD

\subsection{But what about X}
Rebuttal of counter arguments

\section{Conclusion}
Summarizes the argument and offers a final, persuasive thought.

\begin{thebibliography}{00}
\bibitem{b1} A. Piltch, ``Bossware booms as bots determine wether you're doing a good job,'' The Register, Nov. 23, 2025. [Online]. Available: \url{https://www.theregister.com/2025/11/23/bossware_monitor_remote_employees}
\bibitem{b2} J. Clerk Maxwell, A Treatise on Electricity and Magnetism, 3rd ed., vol. 2. Oxford: Clarendon, 1892, pp.68--73.
\bibitem{b3} I. S. Jacobs and C. P. Bean, ``Fine particles, thin films and exchange anisotropy,'' in Magnetism, vol. III, G. T. Rado and H. Suhl, Eds. New York: Academic, 1963, pp. 271--350.
\bibitem{b4} K. Elissa, ``Title of paper if known,'' unpublished.
\bibitem{b5} R. Nicole, ``Title of paper with only first word capitalized,'' J. Name Stand. Abbrev., in press.
\bibitem{b6} Y. Yorozu, M. Hirano, K. Oka, and Y. Tagawa, ``Electron spectroscopy studies on magneto-optical media and plastic substrate interface,'' IEEE Transl. J. Magn. Japan, vol. 2, pp. 740--741, August 1987 [Digests 9th Annual Conf. Magnetics Japan, p. 301, 1982].
\bibitem{b7} M. Young, The Technical Writer's Handbook. Mill Valley, CA: University Science, 1989.
\bibitem{b8} D. P. Kingma and M. Welling, ``Auto-encoding variational Bayes,'' 2013, arXiv:1312.6114. [Online]. Available: https://arxiv.org/abs/1312.6114
\bibitem{b9} S. Liu, ``Wi-Fi Energy Detection Testbed (12MTC),'' 2023, gitHub repository. [Online]. Available: https://github.com/liustone99/Wi-Fi-Energy-Detection-Testbed-12MTC
\bibitem{b10} ``Treatment episode data set: discharges (TEDS-D): concatenated, 2006 to 2009.'' U.S. Department of Health and Human Services, Substance Abuse and Mental Health Services Administration, Office of Applied Studies, August, 2013, DOI:10.3886/ICPSR30122.v2
\end{thebibliography}

\vspace{12pt}
\color{red}
IEEE conference templates contain guidance text for composing and formatting conference papers. Please ensure that all template text is removed from your conference paper prior to submission to the conference. Failure to remove the template text from your paper may result in your paper not being published.

\end{document}
