\documentclass[conference]{IEEEtran}
\IEEEoverridecommandlockouts
% ! LTeX: language=en-US
\usepackage{cite}
\usepackage{url}
\usepackage{amsmath,amssymb,amsfonts}
\usepackage{algorithmic}
\usepackage{graphicx}
\usepackage{textcomp}
\usepackage{xcolor}
\def\BibTeX{{\rm B\kern-.05em{\sc i\kern-.025em b}\kern-.08em
    T\kern-.1667em\lower.7ex\hbox{E}\kern-.125emX}}
\begin{document}

\title{Ethical Arguments for the use of AI-driven Monitoring Systems for Remote Employees}

\author{\IEEEauthorblockN{Lennart Vetter}
\IEEEauthorblockA{\textit{Albert Nerken School of Engineering (The Cooper Union)} \\
New York City, USA \\
lennart.vetter@cooper.edu}}
\maketitle

\begin{abstract}
This paper presents ethical justifications for using AI-driven Monitoring Systems that 
incorporate employee biometric data, arguing that these systems are guided by the moral 
imperative of improving employee welfare, increasing workplace fairness and having a positive business impact.
\\
\end{abstract}

\begin{IEEEkeywords}
Algorithmic Control, Surveillance, Machiavellianism, Authoritarianism, Orwellianism.
\end{IEEEkeywords}

\section{The Need for Employee Monitoring} % INTRO

The recent shift to remote work has provided employees with greater workplace flexibility and access to 
more resources and opportunities. Remote work has effectively increased employee content, saving valuable
time commuting and x (cite study).
However, these advantages come with several unique challenges: One of the primary difficulties 
remote workers face in these online environments is the lack of interaction with their supervisor 
or colleagues. The absence of a personal connection in virtual work environments and the 
blurring of work and non-work boundaries can negatively impact the motivation of employees, 
contributing to a sense of isolation and a lack of motivation. In some cases, the lack of human
supervision even negatively impact the work related safety of employees.
AI-driven Monitoring Systems describe a set of tools which autonomously track biometric data
of employees such as facial expressions, eye movements, head position and body posture as 
well as behavioral data like keystrokes and mouse movements in an effort to grant real time and 
fine-grained insights into workers' productivity and health metrics. 

This systematic review lays out the benefits of implementing AI driven Employee Monitoring Systems.
These systems provide data driven support for employee well-being, increase and enforce employee 
workplace safety, optimize organizational resources and enforce equal accountability benefitting 
both the employer and the remote employee all of which serve as a moral justification for the use 
of these systems.

\section{New Environments, New Challenges} % DISTRACTION / ASSISTANCE

Remote environments differ significantly from traditional office environments. Many factors that 
emanate from the physical work environment such as the presence of coworkers,
adequate work equipment and a dedicated work environment effect workers' productivity and 
are for the most part absent in remote environments. Furthermore, remote environments give rise 
to distractions from both external factors like environmental disturbances and internal 
factors like low levels of engagement.

\subsection{Staying Focused}

These distractions include constant phone notifications, emails, phone calls or other external interruptions. 
In remote work settings, these distractions are of particular concern as the boundary between work related and 
non-work related tasks blurs. Multitasking, such as reacting to and taking care of private errands or interacting 
with social media, significantly reduces cognitive efficiency \cite{Blasiman2018}.
Remote workers are particularly vulnerable to distractions that significantly impact their focus and 
overall performance outcomes. As \cite{Altmann2014} demonstrated, momentary interruptions, even as 
brief as a few seconds, can significantly increase error rates by disrupting the cognitive
processes required to maintain task and sequence continuity. As a result, maintaining concentration 
and staying focused on a task becomes a major challenge in remote work settings. Understanding, 
detecting and predicting when remote workers become distracted is therefore of high importance.

Physiological and biometric signals such as cognitive load, blinking of the eyes, facial expression and 
body pose have proven highly valuable in predicting the current state of the worker. When integrating these 
into analytics tools, a more comprehensive insight into the behavior can be obtained.
Simply by observing the head position, \cite{Becerra2025} was able to predict phone distractions 
with 87\% accuracy. Similarly, \cite{Daza2024} showed that webcam data can be successfully leveraged 
to accurately predict high and low attention levels by tracking eye blinking and facial expressions. 
A different study showed that by simply measuring body posture and facial expression using a simple 
webcam in combination with machine learning was able to predict whether participants are concentrated 
with more than 90\% recall \cite{Betto2022}.

\cite{Kaczorowska2021} showed that assessing the cognitive workload levels using eye tracking data 
yielded great results. Understanding these different performance metrics such as attention levels and 
cognitive workload levels is of great importance in analyzing human mental fatigue and the performance of 
intellectual tasks. Biometric data can further be leveraged to "identify sequential interaction patterns 
associated with moments of distraction" \cite{Rodriguez2023}, offering the unique opportunity to optimize 
the order of sequential tasks to be performed by the remote employee. 

This shows that tracking biometric data is an effective way to measure employee engagement which opens
the door to implement systems that assist employees in their work.
Employee Monitoring Systems act as an assistant to the employee, working with the employee to by 
providing real time objective data and guidance to counteract cognitive challenges like distractions and 
mental fatigue as well as allowing for personalized feedback. Such a system would be able to 
recommend sequential based on the worker's current attention level thus maximizing productivity while 
retaining attention as well as minimizing distractions and supporting the employee in mantaining focus.
Hence, the moral justification arises from the supportive nature of these systems.


\subsection{Improving Workplace Safety} % SAFETY

Assisting the employee in terms of productivity is not the only capability of modern monitoring systems. 
When human supervision is absent, these systems watch out for the employees by utilizing the collected 
biometric and behavioral data to provide real time risk mitigation and personalized workload management. 
Physical safety of employees is the primary ethical and legal obligation of every employer. 
While this might be less of a concern in white collar remote work, blue collar Remote Monitoring Systems for 
industries like the Construction Industry are able to assist with increasing employee safety in various ways.

Real-time location tracking of employees, which is particularly useful in environments with reduced visibility,
can be used to prevent severe injury by avoiding worker collision with heavy machinery or vehicles in 
motion \cite{Arboleya2021}. Moreover, Computer Vision Systems are able to track if workers are utilizing
adequate safety equipment like wearing helmets or masks \cite{Das2022} for the work they are performing. 
Mobility related work heavily benefits from using behavioral and biometric data such as 
eye motion tracking, blink frequency, facial expressions and head movement in combination with Neural 
Networks to predict driver drowsiness with high degrees of accuracy, preventing sleepiness induced 
accidents \cite{Shaik2023}.

By using biometric data, these Remote Monitoring Systems can actively improve safety of employees in 
blue collar work environments and can also ensure that government imposed safety regulations are met.
As it is a company's highest ethical commitment to protect workers and ensure their physical safety, 
the use of these systems is morally justified.

\subsection{Employee Health as a Top Priority} % HEALTH

While safety addresses the workplace as a whole, individual employee health
also benefits tremendously from Remote Monitoring Systems, especially in blue collar work environments.
These systems promise to handle physical demands for each worker in a personalized manner by leveraging
individual biometric data: Wearable technologies can estimate individual activity levels through proxies 
like heart rate, which is a physiological variable that has been extensively used to quantify physical 
demands because of its association with cardiovascular loads \cite{Arias2023}. This essential information 
can then be used to manage physical demands within acceptable limits for each individual. 
This provides personalized alerts to the employee or manager, promoting mandatory pauses or task reallocation 
based on individual physiological limits.
Especially in construction work, work fatigue leads to poor judgement, increased risk of injuries, 
decreased productivity and lower quality of work \cite{Abdelhamid2002}. 
Since physical fatigue is a predominant risk factor for injuries and illnesses in the 
construction industry, it is essential to monitor fatigue to reduce its adverse effects and prevent long-term 
health problems which can be achieved by tracking biometric data like aerobic fatigue threshold \cite{Bangaru2022}.
Monitoring systems are not only useful for real-time applications but also provide data for mitigating long term 
occupational injury risk. By accumulating biometric data from different employees over time, insights can be 
gained in which behavioral patterns predict injury risk and consequently conclusions can be drawn to 
improve long-term workspace safety procedures and requirements \cite{Khairuddin2022}.

Remote Monitoring Systems therefore go well beyond supervision and proactively promote employee health. Improving
employee wellbeing and physical health lays the moral foundation for the use of this technology.

\subsection{Making Informed Decisions} % FAIRNESS AND BUSINESS ASPECTS

Beyond improving workplace safety, preventing employee injuries and preserving their health as well as 
assisting employees in staying focus and avoiding distractions, these systems also benefit the employer side.
Fine-grained performance insights and data allows managers to quickly identify company inefficiencies. Managers
are able to see which tasks employees are currently tackling, how much time has been spent on every task. 
Precisely knowing how current company resources are utilized opens the opportunity to channel them differently 
once inefficiencies are discovered.
Anecdotally, Ivan Petrovic, CEO and Founder of Insightful, told the story of how a client was able to save 
\$2 Million dollars by discovering that none of its employees used an expensive piece of software \cite{Piltch2025}.
Similarly, tracking each remote employee, the employer can quickly determine who is being underutilized 
and who does too much work. By providing objective metrics, Monitoring Systems promotes fairness and 
equality among employees, allowing for an equal balance of tasks and workload across team members which 
adheres to the moral principle of equal employee contribution to the success of the company.

Said insights also allow managers to make informed decisions on who to promote and to whom assign specific, high 
stakes projects based on the performance measures that have been gathered per employee. Leveraging biometric 
data for promotions and assignments fosters a culture of objective and transparent decisions, moving away from
subjective bias and management favoritism.

Remote Employee Productivity Monitoring can also serve a dual purpose of improving cyber-security by using the 
collected biometric and behavioral data such as keystrokes, eye motion, head motion or mouse tracking for 
continuous biometric authentication \cite{Boshoff2025}, flagging unusual or non-work related activities that 
could pose a risk to company data integrity. 

Consequently, the use of Remote Employee Productivity Monitoring can be tremendously beneficial to the employing 
company. If a company is unable to ensure company efficiency and security, it might be forced to end remote 
work policies which would be detrimental to employees who rely on that flexibility. It is therefore necessary 
to ensure the viability of the company as a prerequisite for employment.

The following paragraph shall address a common concern that employees might have when Remote Monitoring Systems are 
used.

\section{The Orwellian Concern}
Privacy focused employees often strongly reject the idea of having AI driven Monitoring Systems track their biometric
data during work hours, often stating that this manifests an unacceptable invasion of privacy and creates a 
monitoring culture that goes well beyond work related data collection. The argument is made that this contradicts 
the moral obligation of data privacy. This valid concern shall subsequently be addressed.

- proposed system will only monitor employees on work devices and during work hours, thus ensuring that data is only
for work related purposes
- collected data is used transparently, the employer communicates the way it is used to the employees, this creates 
trust (cite study)
- collected data is used for the benefit of the employee, increases safety, prevents injuries, increases productivity,
reduces distractions
- remote work contract can only be upheld if employer can be sure of performance. Monitoring Systems grant security.
If no such security is given, then the employer might be forced to revoke remote work policy, negatively impacting
work flexibility for employees.
- Monitoring the digital workspace shall be analogous to a supervisor observing the physical workspace.

\section{Conclusion}

Summarizes the argument and offers a final, persuasive thought. It must rhetorically tie the presented benefits back to the ethical argument promised in the title.

\begin{thebibliography}{00}
\bibitem{Piltch2025} A. Piltch, ``Bossware booms as bots determine wether you're doing a good job,'' The Register, Nov. 23, 2025. [Online]. Available: \url{https://www.theregister.com/2025/11/23/bossware_monitor_remote_employees}
\bibitem{Becerra2025} Alvaro Becerra, Roberto Daza, Ruth Cobos P{\'e}rez, Aythami Morales, Mutlu Cukurova, and Julian Fi{\'e}rrez, ``AI-based Multimodal Biometrics for Detecting Smartphone Distractions: Application to Online Learning,'' {\it ArXiv}, 2025, vol. abs/2506.17364.
\bibitem{Altmann2014} Altmann, Erik M. et al. “Momentary interruptions can derail the train of thought.” Journal of experimental psychology. General 143 1 (2014): 215-26 .
\bibitem{Blasiman2018} Blasiman, Rachael N. et al. “Distracted Students: A Comparison of Multiple Types of Distractions on Learning in Online Lectures.” Scholarship of Teaching and Learning in Psychology 4 (2018): 222–230.
\bibitem{Betto2022} Betto, Iku et al. “Distraction detection of lectures in e-learning using machine learning based on human facial features and postural information.” Artificial Life and Robotics 28 (2022): 166 - 174.
\bibitem{Daza2024} Daza, Roberto et al. “DeepFace-Attention: Multimodal Face Biometrics for Attention Estimation With Application to e-Learning.” IEEE Access 12 (2024): 111343-111359.
\bibitem{Rodriguez2023} Rodríguez, Andrés Ovidio Restrepo et al. “Application of learning analytics for sequential patterns detection associated with moments of distraction in students in e‐learning platforms.” Computer Applications in Engineering Education 32 (2023): n. pag.
\bibitem{Kaczorowska2021} Kaczorowska, Monika et al. “Interpretable Machine Learning Models for Three-Way Classification of Cognitive Workload Levels for Eye-Tracking Features.” Brain Sciences 11 (2021): n. pag.
\bibitem{Arboleya2021} A. Arboleya, J. Laviada, Y. Álvarez-López and F. Las-Heras, "Real-Time Tracking System Based on RFID to Prevent Worker–Vehicle Accidents," in IEEE Antennas and Wireless Propagation Letters, vol. 20, no. 9, pp. 1794-1798, Sept. 2021, doi: 10.1109/LAWP.2021.3097136. keywords: {Safety;Head;Linear antenna arrays;RFID tags;Antenna radiation patterns;Receiving antennas;Radar tracking;Industrial environments;radio frequency identification (RFID);receiver signal strength (RSS);real-time tracking},
\bibitem{Arias2023} Arias O, Groehler J, Wolff M, Choi SD. Assessment of Musculoskeletal Pain and Physical Demands Using a Wearable Smartwatch Heart Monitor among Precast Concrete Construction Workers: A Field Case Study. Applied Sciences. 2023; 13(4):2347. https://doi.org/10.3390/app13042347
\bibitem{Khairuddin2022} Khairuddin, Mohamed Zul Fadhli et al. “Occupational Injury Risk Mitigation: Machine Learning Approach and Feature Optimization for Smart Workplace Surveillance.” International journal of environmental research and public health vol. 19,21 13962. 27 Oct. 2022, doi:10.3390/ijerph192113962
\bibitem{Bangaru2022} Bangaru, Wang, Aghazadeh (2022). Automated and Continuous Fatigue Monitoring in Construction Workers Using Forearm EMG and IMU Wearable Sensors and Recurrent Neural Network. Sensors, 22(24), 9729. https://doi.org/10.3390/s22249729
\bibitem{Das2022} Das Prangon, Abhi Sarafat, Suhreed F., Uddin Jalal, Jahan Shahriar, Lara Roknuzzaman, Kashem Md. (2022). An Intelligent Industrial Safety and Health Monitoring System for Industry 4.0. 
\bibitem{Shaik2023} Md. Ebrahim Shaik, "A systematic review on detection and prediction of driver drowsiness", Transportation Research Interdisciplinary Perspectives 21 (2023), ISSN 2590-1982, https://doi.org/10.1016/j.trip.2023.100864.
\bibitem{Abdelhamid2002} Abdelhamid, T. S., and J. G. Everett. (2002). Physiological Demands during Construction Work. \textit{Journal of Construction Engineering and Management}, 128(5), 427--437.
\bibitem{Boshoff2025} Dutliff Boshoff, Gerhard P. Hancke, "A classifications framework for continuous biometric authentication (2018–2024)", Computers \& Security V. 150 (2025), https://doi.org/10.1016/j.cose.2024.104285.
\end{thebibliography}

\vspace{12pt}

\end{document}
