\documentclass[conference]{IEEEtran}
\IEEEoverridecommandlockouts

\usepackage{cite}
\usepackage{url}
\usepackage{amsmath,amssymb,amsfonts}
\usepackage{algorithmic}
\usepackage{graphicx}
\usepackage{textcomp}
\usepackage{xcolor}
\def\BibTeX{{\rm B\kern-.05em{\sc i\kern-.025em b}\kern-.08em
    T\kern-.1667em\lower.7ex\hbox{E}\kern-.125emX}}
\begin{document}

\title{Ethical Arguments for the use of AI-driven productivity monitoring of remote employees}

\author{\IEEEauthorblockN{Lennart Vetter}
\IEEEauthorblockA{\textit{Albert Nerken School of Engineering (The Cooper Union)} \\
New York City, USA \\
lennart.vetter@cooper.edu}}
\maketitle

\begin{abstract}
This paper highlights the effectiveness and the positive business impact of monitoring the productivity of remote workers using biometric data like facial expressions, body posture and eye movement as well as behavioral data like key strokes and mouse movements. 
The effective utilization of this data in combination with modern AI systems yields promosing improvement in productivity among remote employees, while increasesing cyber security and granting fine grained control over task distribution to the employer.
\end{abstract}

\begin{IEEEkeywords}
Algorithmic Control, Surveillance, Machiavellianism, Authoritarianism, Orwellianism.
\end{IEEEkeywords}

\section{The Need for Employee Monitoring} % INTRO

The recent shift to remote work has provided employees with greater flexibility and access to 
more resources and opportunities. 
However, these advantages come with several unique challanges: One of the primary difficulties 
remote workers face in these online environments is the lack of interaction with their supervisor 
or collegues. The absence of this personal connection in 
virtual work environments and the blurring of work and non-work boundaries can negatively impact 
the motivation of employees, contributing to a sense of isolation and a lack of motivation.
Moreover, the lack of supervision might negatively impact the safety of employees in their work 
environment.
AI-driven Monitoring Systems describe a set of tools which autonomously track biometric data
of employees such as facial expressions, eye movements, head position and body posture as 
well as behavioral data like key strokes and mouse movements in an effort to grant real time and 
fine grained insights into workers' productivity and health metrics. 

This systematic review lays out the benefits of implementing AI driven Employee Monitoring Systems.
These systems provide data driven support for employee well-being, increase and enforce employee 
workpalce safety, optimize organizational resources and enforce equal accountability benefitting 
both the employer and the remote employee.

Add some section about work ethics and what can be done from employer perspective with respect to workers

\section{New Environments, New Challenges} % DISTRACTION / ASSISTANCE

Remote environments differ significantly from traditional office environments. Many factors that 
emanate from the physical work environment such as the presence of coworkers and 
adequat work equipment effect workers' productivity and are for the most part absent in remote environments.
Furthermore, remote environment give rise to distractions from both external factors like 
environmental disturbances as well as internal factors like low levels of engagement.

These distractions include constant phone notifications, emails, phone calls or other external interruptions. 
In remote work settings, these distractions are of particular concern as the boundary between work related and 
non work related tasks blurs. Multitasking, such as reacting to taking care of privat errands or interacting 
with social media, significantly reduces cognitive efficiency \cite{Blasiman2018}.
Remote workers are particularily vulnerable to distractions that significantly impact their focus and 
overall performance outcomes. As \cite{Altmann2014} demonstrated, momentary interruptions, even as 
brief as a few seconds, can significantly increase error rates by discrupting the cognitive
processes required to maintain task and sequence continuity. As a result, maintaining concentration 
and staying focused on a task becomes a major challenge in remote work settings. Understanding, 
detecting and predicting when remote workers become distracted is therefore of high importance.

Physiological and biometric signals such as cognitive load, blinking of the eyes, facial expression and 
body pose have proven highly valuable in predicting the current state of the worker. When integrating these 
into analytics tools, a more comprehensive insight into the behavior can be obtained.
Simply by observing the head position \cite{Becerra2025} was able to predict phone distractions 
with 87\% accuracy. Similarily, \cite{Daza2024} showed that webcam data can be sucesfully leveraged 
to accuracte predict high and low attention levels by tracking Eyeblink and Facial Expressions. 
A different study showed that by simply measuring body posture and facial expression using a simple 
webcam in combination with machine learning was able to predict wether or not participants are concentrated 
with more than 90\% recall \cite{Betto2022}.

\cite{Kaczorowska2021} showed that assessing the cognitive workload levels using eye tracking data 
yielded great results. Understadning these different performance metrics such as attention levels and 
cognitive workload levels is of great importance in analysing human mental fatigue and the performance of 
intellectual tasks. Biometric data can further be leveraged to "identify sequential interaction patterns 
associated with moments of distraction" \cite{Rodriguez2023}, offering the unique opportunity to optimize 
the order of sequential tasks to be performed by the remote employee. Sequential tasks could be 
recommending based on the worker's current attention level thus maximizing productivity while retaining attention. 

Emloyee Monitoring Systems act as an assistant to the employee, working with the employee to by 
providing real time objective data and guidance to counteract cognitive challanges like distractions and 
mental fatigue as well as allowing for personalized feedback.

\subsection{Improving Workplace Safety} % SAFETY

Assisting the employee in terms of productivity is not the only capability of modern monitoring systems. 
When human supervision is absent, these systems watch out for the employees by utilizing the collected 
biometric and behavioral data to provide real time risk mitigration and personalized workload management. 
Physical safety of employees is the number one key priority of every work environment. While this might be less of a 
concern white collar remote work, blue collar Remote Monitoring Systems for industries like the construction 
industry are able to assist with increasing employee saftety in various ways.

Real-time location tracking of employees, which is particularily useful in environements with reduced visibility,
can be used to prevent severe injury by avoiding worker collision with heavy machinery or vehicles in motion \cite{Arboleya2021}.
Monitoring workers using Artificial Intelligence to ensure they are adhering to work place safety regulations like 
wearing helmets, masks or other required safety equipment to perform their work increases their own saftey and 
upholds workplace safety regulations \cite{Das2022}.
Mobility related work heavily benefits from using behavioral and biometric data such as 
eye tracking, blink frequency, facial expressions and head movement in combination with Neural Networks or CNNs to 
predict driver drowsiness with high degrees of accuracy, preventing sleepiness induced accidents \cite{Shaik2023}.

By using biometric data, these Employee Monitoring Tools can actively improve safety of employees in blue collar work 
environemnts and ensure that government imposed safety regulations are met.

\subsection{Employee Health as a Top Priority} % HEALTH

While safety adressess the workplace as a whole and varios other stakeholders involved, individual employee health
also benefits tremendously from Remote Employee Monitoring Systems especially in blue collar work environments.
Physical demands for each worker can be handled in a personalized manner by leveraging
individual biometric data: Wearable technologies can estimate indivual activity levels through proxies like heart rate,
which is a physiological variable that has been extensively used to quantify physical demands because of its 
association with cardiovascular loads \cite{Arias2023}. This essential information can then be used to manage physical
demands within acceptable limits for each individual. This provides personalized alerts to the employee or
manager, promoting mandatory pauses or task reallocation based on individual physiological limits.
Especially in construction work, work fatigue leads to poor judgement, increased risk of injuries, 
decreased productivity and lower quality of work \cite{Abdelhamid2002}. 
Since physical fatigue is a predominant risk factor for injuries and illnesses in the 
construction industry, it is essential to monitor fatigue to reduce its adverse effects and prevent long-term 
health problems. Monitoring biometric data like aerobic fatigue treshold to monitor the fatigue level. \cite{Bangaru2022}

Monitoring systems are not only useful for real-time applications but also provide data for minigating long term 
occupational injury risk. By accumulating biometric data from different employees over time, insights can be 
gained in which behavioral patterns predict injury risk and consequently conclusions can be drawn to 
improve longterm workspace safety proceedures and requirements \cite{Khairuddin2022}.

Remote Employee Monitoring Systems therefore go well beyond supervision and proactively promote employee health.

\subsection{Making Informed Decisions} % FAIRNESS AND BUSINESS ASPECTS
- Using pure biometric signals such as blinking of th eyes, eye tracking,  data is free from bias as these signals are purely performance based and do not carry an 
inherent bias.
- Fine grained insights/data allows managers to quickly identify whio is underutilized or overloaded, which helps to
balance tasks acress team members and ensures equality of contribution. 
- Allows managers to make informed decisions on who to promote/ give which projects
- Allows managers to identify non utilized software: Efficient and exhaustive resource utilisation is foundational for sustained business. Obtained insights into worker productivity metrics can subsequently be used to make informed business decisions. The obtained data offer finegrained insights into which tasks
employees are currently tackling, how much time has been spend on every taks. Precisely knowing how current company resources are utilized opens the opportunity to channel them differently once inefficiencies are discovered. Anectodaly, Ivan Petrovic, CEO and Founder of insughtful, 
told the story of how a client was able to save \$2 Milion dollars by discovering that non of it's eployees used an expensive piece of software \cite{Piltch2025}.
Similarily, tracking each remote employee, the employer can quickly determine who is being underutilized and who does too much work. This helps to helps to balance tasks across
team members, ensuring equality among employees so that everyone contributes equally to the success of the company.
- Improves security and compliance bu flagging unusual or non work related activity that could pose a risk to
company data integrety

If a company is unable to ensure company efficiency and security, it might be forced to end remote work policies which 
would be detremental to employees who rely on that flexibility.

This is an ethical approach as it provides a transparent framework for employment decisions, moving beyornd subjective
seat time or management favorism. It is also necessary to ensure the viabilityz ant the organization which is a
prerequisite for employement.

\subsection{The Orweillian Concern}
Remote Employee Monitoring Tools manifest an unacceptable invasion of privacy and create a monitoring culture that goes 
well beyond work related data collection.
This concern is easily adressed as monitoring will be limited to company devices and dedicated work hours only. 
Employees implicitly consent to this kind of monitoring when accepting remote work flexibility. In the end, monitoring 
the digital workspace is analogous to a supervisor observing the physical workspace.

\section{Conclusion}
Summarizes the argument and offers a final, persuasive thought.

\begin{thebibliography}{00}
\bibitem{Piltch2025} A. Piltch, ``Bossware booms as bots determine wether you're doing a good job,'' The Register, Nov. 23, 2025. [Online]. Available: \url{https://www.theregister.com/2025/11/23/bossware_monitor_remote_employees}
\bibitem{Becerra2025} Alvaro Becerra, Roberto Daza, Ruth Cobos P{\'e}rez, Aythami Morales, Mutlu Cukurova, and Julian Fi{\'e}rrez, ``AI-based Multimodal Biometrics for Detecting Smartphone Distractions: Application to Online Learning,'' {\it ArXiv}, 2025, vol. abs/2506.17364.
\bibitem{Altmann2014} Altmann, Erik M. et al. “Momentary interruptions can derail the train of thought.” Journal of experimental psychology. General 143 1 (2014): 215-26 .
\bibitem{Blasiman2018} Blasiman, Rachael N. et al. “Distracted Students: A Comparison of Multiple Types of Distractions on Learning in Online Lectures.” Scholarship of Teaching and Learning in Psychology 4 (2018): 222–230.
\bibitem{Betto2022} Betto, Iku et al. “Distraction detection of lectures in e-learning using machine learning based on human facial features and postural information.” Artificial Life and Robotics 28 (2022): 166 - 174.
\bibitem{Daza2024} Daza, Roberto et al. “DeepFace-Attention: Multimodal Face Biometrics for Attention Estimation With Application to e-Learning.” IEEE Access 12 (2024): 111343-111359.
\bibitem{Rodriguez2023} Rodríguez, Andrés Ovidio Restrepo et al. “Application of learning analytics for sequential patterns detection associated with moments of distraction in students in e‐learning platforms.” Computer Applications in Engineering Education 32 (2023): n. pag.
\bibitem{Kaczorowska2021} Kaczorowska, Monika et al. “Interpretable Machine Learning Models for Three-Way Classification of Cognitive Workload Levels for Eye-Tracking Features.” Brain Sciences 11 (2021): n. pag.
\bibitem{Arboleya2021} A. Arboleya, J. Laviada, Y. Álvarez-López and F. Las-Heras, "Real-Time Tracking System Based on RFID to Prevent Worker–Vehicle Accidents," in IEEE Antennas and Wireless Propagation Letters, vol. 20, no. 9, pp. 1794-1798, Sept. 2021, doi: 10.1109/LAWP.2021.3097136. keywords: {Safety;Head;Linear antenna arrays;RFID tags;Antenna radiation patterns;Receiving antennas;Radar tracking;Industrial environments;radio frequency identification (RFID);receiver signal strength (RSS);real-time tracking},
\bibitem{Arias2023} Arias O, Groehler J, Wolff M, Choi SD. Assessment of Musculoskeletal Pain and Physical Demands Using a Wearable Smartwatch Heart Monitor among Precast Concrete Construction Workers: A Field Case Study. Applied Sciences. 2023; 13(4):2347. https://doi.org/10.3390/app13042347
\bibitem{Khairuddin2022} Khairuddin, Mohamed Zul Fadhli et al. “Occupational Injury Risk Mitigation: Machine Learning Approach and Feature Optimization for Smart Workplace Surveillance.” International journal of environmental research and public health vol. 19,21 13962. 27 Oct. 2022, doi:10.3390/ijerph192113962
\bibitem{Bangaru2022} Bangaru, Wang, Aghazadeh (2022). Automated and Continuous Fatigue Monitoring in Construction Workers Using Forearm EMG and IMU Wearable Sensors and Recurrent Neural Network. Sensors, 22(24), 9729. https://doi.org/10.3390/s22249729
\bibitem{Das2022} Das Prangon, Abhi Sarafat, Suhreed F., Uddin Jalal, Jahan Shahriar, Lara Roknuzzaman, Kashem Md. (2022). An Intelligent Industrial Safety and Health Monitoring System for Industry 4.0. 
\bibitem{Shaik2023} Md. Ebrahim Shaik, "A systematic review on detection and prediction of driver drowsiness", Transportation Research Interdisciplinary Perspectives 21 (2023), ISSN 2590-1982, https://doi.org/10.1016/j.trip.2023.100864.
\bibitem{Abdelhamid2002} Abdelhamid, T. S., and J. G. Everett. (2002). Physiological Demands during Construction Work. \textit{Journal of Construction Engineering and Management}, 128(5), 427--437.
\end{thebibliography}

\vspace{12pt}

\end{document}
